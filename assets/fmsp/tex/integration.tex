\documentclass[10pt]{beamer}
%\documentclass[handout,10pt]{beamer}
%\mode<presentation>
%{
%  \usetheme{Berkeley}
%  \usecolortheme{seahorse}
%  \usefonttheme{default}
%  \setbeamertemplate{navigation symbols}{}
%  \setbeamertemplate{caption}[numbered]
%}

\usetheme{metropolis}

\usepackage[english]{babel}
\usepackage[utf8x]{inputenc}
\usepackage{caption}
\usepackage{multirow}
\usepackage{mathrsfs}
\usepackage{graphicx}
\usepackage{amsmath}
\usepackage{graphicx}
\usepackage[compatibility=false]{caption}
\usepackage{subcaption}
\usepackage[normalem]{ulem}
\DeclareMathOperator{\tr}{tr}
\usepackage{textpos}
\usepackage{animate}
\DeclareMathOperator{\sech}{sech}
\DeclareMathOperator{\cosech}{cosech}
\DeclareMathOperator{\cosec}{cosec}

\title[FMSP Further Mathematics]{Integration}
%\titlegraphic{\includegraphics[height=1.57cm]{logo.jpg}}
\author[Scott Morgan]{\textbf{Scott Morgan}}
\institute{\textit{Further Mathematics Support Programme - WJEC A-Level Further Mathematics} \\
\textit{31st March 2018}
\\ \\ \\
\textit{scott3142.com | @Scott3142}}
\date

\begin{document}

\begin{frame}
  \maketitle
\end{frame}

\begin{frame}{Topics}
  \begin{itemize}
	  \item Integration by completing the square
 	  \item Integration by substitution
  \end{itemize}
\end{frame}

\begin{frame}{Topics}
  \begin{itemize}
	  \item Integration by completing the square
  \end{itemize}
\end{frame}

\begin{frame}{Completing the Square}

	\textbf{Examples:}
	
	Complete the square for the following quadratics:
	
	\begin{align*}
		x^2 + 6x + 1 \\
		5 - 8x - x^2 \\
		2x^2 - 8x + 11 
	\end{align*}
	
\end{frame}

\begin{frame}{Integration by Completing the Square}

	\textbf{Examples:}
	
	Find
	
	\begin{align*}
		\int \frac{1}{\sqrt{x^2 + 2x + 2}} dx \\
		\int \frac{1}{x^2 + 5x + 7} dx \\
		\int \frac{1}{1-4x-2x^2} dx
	\end{align*}
	
\end{frame}

\begin{frame}{Integration by Completing the Square}

	\textbf{Examples:}
	
	Find
	
	\begin{align*}
		\int \frac{1}{\sqrt{x^2 - 2x + 3}} dx \\
		\int \frac{1}{\sqrt{1 - 4x - x^2}} dx \\
		\int \frac{1}{\sqrt{x^2 - 2x - 5}} dx \\
		\int \frac{1}{\sqrt{2x^2 + 5x + 1}} dx \\
		\int \frac{1}{\sqrt{3 - x - x^2}} dx
	\end{align*}
	
\end{frame}

\begin{frame}{Topics}
  \begin{itemize}
	  \item Integration by substitution
  \end{itemize}
\end{frame}

\begin{frame}{Trigonometric Substitutions}

	Expressions of the form $\sqrt{a^2-x^2}$ can be reduced to the square root of a single term by a substitution either of the form
	
	\begin{equation*}
		x = a\sin(\theta)
	\end{equation*}
	
	or of the form

	\begin{equation*}
		x = a\cos(\theta)
	\end{equation*}	
	
\end{frame}

\begin{frame}{Integration by Completing the Square}

	\textbf{Examples:}
	
	By making a substitution write the following as a single trigonometric term in terms of $\theta$
	
	\begin{align*}
		\sqrt{9-x^2} \\
		\sqrt{25-x^2} \\
		\sqrt{1-4x^2} \\
		\sqrt{4-9x^2} \\
		\sqrt{25-16x^2}
	\end{align*}
	
\end{frame}

\begin{frame}{Trigonometric Substitutions}

	Expressions of the form $\sqrt{a^2-x^2}$ can be reduced to the square root of a single term by a substitution of the form
	
	\begin{equation*}
		x = a\sinh(\theta)
	\end{equation*}
	
	Expressions of the form $\sqrt{x^2-a^2}$ can be reduced to the square root of a single term by a substitution of the form
		
	\begin{equation*}
		x = a\cosh(\theta)
	\end{equation*}
	
\end{frame}

\begin{frame}{Integration by Completing the Square}

	\textbf{Examples:}
	
	By making a substitution write the following as a single trigonometric term in terms of $\theta$
	
	\begin{align*}
		\sqrt{x^2-25} \\
		\sqrt{x^2+16} \\
		\sqrt{4x^2-1} \\
		\sqrt{25x^2-1} \\
		\sqrt{9x^2 - 4} \\
		\sqrt{4x^2 + 25}
	\end{align*}
	
\end{frame}

\begin{frame}{Integration by Completing the Square}

	\textbf{Example:}
	
	By making the substitution $x = 2\cosh(u)$, find
	
	\begin{equation*}
		\int \frac{x^2}{\sqrt{x^2-4}} dx
	\end{equation*}

\end{frame}

\begin{frame}{Integration by Completing the Square}

	\textbf{Example:}
	
	By making the substitution $x = \sinh(u)$, find
	
	\begin{equation*}
		\int (x^2 + 1)^{-\frac{3}{2}} dx
	\end{equation*}

\end{frame}

\begin{frame}{Integration by Completing the Square}

	\textbf{Exam Question:}
	
	\begin{itemize}
	
		\item [(a)] Assuming the derivatives of $\sinh(\theta)$ and $\cosh(\theta)$, use the quotient rule to find the derivative of $\coth(\theta)$ \hfill[3]
		
		\item [(b)] Use the substitution $x = \cosh(\theta)-2$ to evaluate the integral
			
			\begin{equation*}
				\int_1^2 \frac{dx}{\left(x^2+4x+3\right)^{\frac32}}
			\end{equation*}		
			
		giving your answer correct to three significant figures. \hfill[8]

		
	\end{itemize}

\end{frame}

\begin{frame}{Integration by Completing the Square}

	\textbf{Exam Question:}
	
	\begin{itemize}
	
		\item [(a)] By expressing $\sech(x)$ and $\tanh(x)$in terms of exponential functions, show that 
		
			\begin{equation*}
				\sech^2(x) + \tanh^2(x) = 1
			\end{equation*}
		
		\item [(b)] Use the substitution $x = \sinh(u)$ to evaluate the integral
			
			\begin{equation*}
				\int_0^1 \frac{x^2}{\left(1+x^2\right)^{\frac32}}dx
			\end{equation*}		
			
		giving your answer correct to three significant figures.

		
	\end{itemize}

\end{frame}

\end{document}
