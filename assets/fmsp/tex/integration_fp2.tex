\documentclass[10pt]{beamer}
%\documentclass[handout,10pt]{beamer}
%\mode<presentation>
%{
%  \usetheme{Berkeley}
%  \usecolortheme{seahorse}
%  \usefonttheme{default}
%  \setbeamertemplate{navigation symbols}{}
%  \setbeamertemplate{caption}[numbered]
%}

\usetheme{metropolis}

\usepackage[english]{babel}
\usepackage[utf8x]{inputenc}
\usepackage{caption}
\usepackage{multirow}
\usepackage{mathrsfs}
\usepackage{graphicx}
\usepackage{amsmath}
\usepackage{graphicx}
\usepackage[compatibility=false]{caption}
\usepackage{subcaption}
\usepackage[normalem]{ulem}
\DeclareMathOperator{\tr}{tr}
\usepackage{textpos}
\usepackage{animate}
\DeclareMathOperator{\sech}{sech}
\DeclareMathOperator{\cosech}{cosech}
\DeclareMathOperator{\cosec}{cosec}

\title[FMSP Further Mathematics]{Integration}
%\titlegraphic{\includegraphics[height=1.57cm]{logo.jpg}}
\author[Scott Morgan]{\textbf{Scott Morgan}}
\institute{\textit{Further Mathematics Support Programme - WJEC A-Level Further Mathematics} \\
\textit{21st April 2018}
\\ \\ \\
\textit{scott3142.com | @Scott3142}}
\date

\begin{document}

\begin{frame}
  \maketitle
\end{frame}

\begin{frame}{Topics}
  \begin{itemize}
	  \item Integration by partial fractions
 	  \item Integration by substitution
  \end{itemize}
\end{frame}

\begin{frame}{Topics}
  \begin{itemize}
	  \item Integration by partial fractions
  \end{itemize}
\end{frame}

\begin{frame}{Lesson Objectives}

	\textbf{To be able to represent the following algebraic functions as partial fractions:}
	
	\begin{equation*}
		\frac{2x+3}{(x^2-1)(x+3)},  \hspace{5mm}
		\frac{x^2+3x-1}{x^2(x+3)},  \hspace{5mm}
		\frac{x}{(x+4)(x+1)^2},  \hspace{5mm}
		\frac{x^2+3x}{x+4} 
	\end{equation*}

	\textbf{To use partial fractions to evaluate integrals such as:}
	
	\begin{equation*}
	\int \frac{1}{x^2+2x-3}dx,  \hspace{5mm}
	\int_1^2 \frac{x+1}{x^3+2x^2}dx
	\end{equation*}
	
\end{frame}

\begin{frame}{Partial Fractions}

\textbf{Examples}

\begin{align*}
\frac{2x+3}{(x^2-1)(x+3)} &= \only<2->{-\frac{1}{4(x+1)}-\frac{3}{8(x+3)}+\frac{5}{8(x-1)}} \\
\frac{x^2+3x-1}{x^2(x+3)} &= \only<3->{-\frac{1}{3x^2}-\frac{1}{9(x+3)}+\frac{10}{9x}} \\
\frac{x}{(x+4)(x+1)^2} &= \only<4->{-\frac{4}{9(x+4)}+\frac{4}{9(x+1)}-\frac{1}{3(x+1)^2}} \\
\frac{x^2+3x}{x+4} &= \only<5->{(x-1)+\frac{4}{x+4}} \\
\end{align*}

\end{frame}

\begin{frame}{Integration with Partial Fractions}

\textbf{Examples}
	
\begin{align*}
\int \frac{1}{x^2+2x-3}dx &= \only<2->{\int \left(\frac{1}{4(x-1)}-\frac{1}{4(x+3)}\right)dx} \\ \only<3->{&= \frac14\left(\ln(1-x)-\ln(x+3)\right) + \text{const.}}\\ \\
\int_1^2 \frac{x+1}{x^3+2x^2}dx &= \only<4->{\int_1^2 \left(\frac{1}{2x^2}-\frac{1}{4(x+2)} + \frac{1}{4x}\right)dx} \\ \only<5->{&= \left[-\frac{1}{2x} + \frac14\ln(x) - \frac14\ln(x+2)\right]_1^2} \\ \only<6->{&\approx 0.35137}
\end{align*}

\end{frame}

\begin{frame}{Topics}
\begin{itemize}
	\item Integration by substitution
\end{itemize}
\end{frame}

\begin{frame}{Trigonometric Substitutions}

Expressions of the form $\sqrt{a^2-x^2}$ can be reduced to the square root of a single term by a substitution either of the form

\begin{equation*}
x = a\sin(\theta)
\end{equation*}

or of the form

\begin{equation*}
x = a\cos(\theta)
\end{equation*}	

\end{frame}

\begin{frame}{Integration by Substitution}

\textbf{Examples:}

By making a substitution write the following as a single trigonometric term in terms of $\theta$

\begin{align*}
\sqrt{9-x^2} \\
\sqrt{25-x^2} \\
\sqrt{1-4x^2} \\
\sqrt{4-9x^2} \\
\sqrt{25-16x^2}
\end{align*}

\end{frame}

\begin{frame}{Integration by Substitution}

\textbf{Examples:}

Solve the following integration problems:

\begin{align*}
\int \frac{1}{\sqrt{9-x^2}} \\ \\
\int \frac{1}{\sqrt{25-x^2}} \\ \\
\int \frac{2x+3}{x^2+4} 
\end{align*}

\end{frame}

\begin{frame}{Integration by Substitution}

\textbf{Examples:}

Use the substitution $y = x^2$ to evaluate the integral:

\begin{equation*}
\int_1^4 \frac{dx}{\sqrt{x(9-x)}}
\end{equation*}

giving your answer correct to two significant figures.
\end{frame}

\begin{frame}{Integration by Substitution}

\textbf{Examples:}

Use the substitution $u = x^{\frac32}$ to evaluate the integral:

\begin{equation*}
\int_1^4 \frac{\sqrt{x}}{1+x^3}dx
\end{equation*}

giving your answer correct to two significant figures.
\end{frame}

\begin{frame}{Integration with Variable Limits}

\textbf{Two formulae:}

\begin{align*}
\frac{d}{dx}\int_a^x f(t)dt &= f(x) \\ \\
\frac{d}{dx}\int_a^{g(x)} f(t)dt &= f(g(x))g'(x)
\end{align*}

\textbf{Examples:}

Evaluate:

\begin{equation*}
\frac{d}{dx}\left(\int_1^{x^2} \sin\left(\frac1t\right)dt\right)
\end{equation*}

\textit{Hint: Make a substitution $u = x^2$}

\end{frame}

\end{document}
